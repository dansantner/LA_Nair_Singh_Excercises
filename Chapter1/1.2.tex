\documentclass{article}
\usepackage{amsmath}
\usepackage{amsfonts} 
\usepackage{graphicx}
\title{Excercises for Sect. 1.2}
\author{Student}
\begin{document}
\maketitle
\section*{2.}
The constraints $a+b+c=\alpha$ and $x+x=2x$ would require $2 \alpha = \alpha$.  The only way this can be valid is if $\alpha = 0$.
\section*{3.}
$V=\mathbb{Z}^2, \mathbb{F} = \mathbb{R}$\\
Closed under addition and additive inverse. e.g. $(a,b) + (c,d) = (a +c, b + d)$ and $(a,b) + (a, b) = 0$ are valid for integers.\\ 
Not closed under scalar multiplication. e.g. $\sqrt{2}x$
\section*{4.}
$V=\mathbb{R}^2,U=\{(a,b): a=0, b\ne0$ or $a \ne 0, b=0\}$\\
Not closed under addition: e.g. (1,0) + (0,1) = (1,1)
\section*{8.}
This is true.  We are simply removing elements of the  original spanning set of $P(\mathbb{R})^m$ which is still a subspace.\\
Ex: $1 + x^5$.\\
\end{document}