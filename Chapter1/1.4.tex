\documentclass{article}
\usepackage{amsmath}
\usepackage{amsfonts} 
\usepackage{graphicx}
\title{Excercises for Sect. 1.4}
\author{Student}
\begin{document}
\maketitle
\section*{1.}
a-g are rref calculations.
h) a = 2(b + c) so not independent.
i and j are independent.  Used in standard ODE solutions.
k is a fourier expansion so independent.
\section*{5.}
By counterexample ${e_1,e_2,e_3,e_4}$ is LI in $\mathbb{R}^4$ but ${(1,1,0,0),(0,1,1,0),(0,0,1,1),(1,0,0,1)}$ is not.
\section*{7.}
$\alpha(a,b)+\beta(c,d)=0$\\
$\alpha a + \beta c = 0, \alpha b + \beta d = 0$\\
reduces to $\beta bc = \beta ad$. The only way for $\beta = 0$ to be the only solution is if $bc \ne ad$.
\section*{8.}
a) Yes. The dependence of two elements in one of the original sets is still dependent.\\
b) No. See question 5.\\
c) Not necessarily. Intersection could be 0.
d) Yes. If both sets are comprised of LI elements, any intersection must be only those LI elements they have in common.
\section*{13.}
??
\end{document}